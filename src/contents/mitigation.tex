%! Author = opiwa
%! Date = 04.01.2023

\subsection{Mitigation}\label{subsec:mitigation}
Da es sich bei Log4Shell um einen Zero-day handelt, war es besonders wichtig in der Zeit bis zum ersten Patch Möglichkeiten der Mitigierung zu finden.

Eine der schnellsten Möglichkeiten Log4Shell zu mitigieren, ist es\newline \verb|org.apache.logging.log4j.core.lookup.JndiLookup|\newline
aus dem classpath zu entfernen.\footfullcite{redditThread}
Dieser Schritt führt dazu, dass der Logger keine \gls{jndi} Lookups mehr durchführen kann, wodurch der Angriffsvektor komplett wegfällt.
Da \verb|JndiLookup| allerdings ein wesentlicher Bestandteil des Featuresets von Log4j 2 ist, ist dieser Schritt (obwohl effektiv) nur eine Kurzzeitlösung.

Das Problem dieser Lösung ist allerdings, dass dadurch nur die \gls{rce} Sicherheitslücken in CVE-2021-44228 mitigiert wird.
Es ist allerdings weiterhin möglich, dass Benutzereingaben evaluiert werden und zu StackOverflows führen oder private Daten preisgeben.\footfullcite{log4jSecurity}
Daher ist es trotz der Mitigierungsmöglichkeiten wichtig, die Sicherheitspatches so früh wie möglich aufzuspielen.
