%! Author = johannes
%! Date = 03.01.23

\subsection{Lightweight Directory Access Protocol}\label{subsec:ldap}
%\begin{wrapfigure}{l}{0.33\textwidth}
%    \begin{center}
%        \includegraphics[width=0.3\textwidth]{images/ldap-logo}
%    \end{center}
%    \source{\href{https://ldap.com/}{LDAP Website}}
%    \caption{LDAP Logo}
%\end{wrapfigure}
\gls{ldap} ist ein Netzwerkprotokoll, das in RFC 4511\footcite{rfc4511} spezifiziert ist.
Ein \glsaccessshort{ldap} Directory Server ist ein standardisierter Weg, Datenbank Anfragen zu codieren und kann eine klassische \gls{api} ersetzen.
Das \glsaccessshort{ldap} Protokoll kann zunächst auf zwei verschiedene Arten verwendet werden: als Suche und als Lookup.
Eine Suche wie \verb|ldap://127.0.0.1:1389/o=JavaObject| findet ein Java Objekt und gibt dessen Attribute zurück.
Ein Lookup wie \verb|ldap://127.0.0.1:1389/JavaObject| hingegen gibt ein benanntes Java Objekt direkt zurück.
