%! Author = chaorn
%! Date = 06.01.23
\subsubsection{JNDI Architektur und Funktionsweise}

Um das zugrunde liegende Problem besser verstehen zu können, hilft ein etwas tieferer Einblick in die Architektur
und der grundlegenden Funktionalität der \gls{jndi} Bibliothek.
\begin{figure}[!htb]\label{fig:jndiarch} % ! - override default, h - place here, t - place figure at top of a page, b - place figure at bottom of a page
    \begin{center}
        \includegraphics[scale=0.75]{images/jndiarch}
    \end{center}
    \caption{JNDI Architektur}
\end{figure}
\bigskip

Das \gls{jndi} besteht im Grunde aus einer \gls{api} und einem \gls{spi}.
Der Entwickler einer Java Applikation interagiert überwiegend mit der \gls{api}, welche sogenannte \textit{naming} und \textit{directory services} zur Verfügung stellt.

Für spezielle Funktionalitäten sorgt das \gls{spi}, das je nach Bedarf ein benötigtes Modul - ähnlich wie ein Plugin-Manager - aktiviert.\footcite{JNDIArchitektur}
Zu den bereitgestellten Services gehören unter anderem \gls{ldap} und \gls{rmi}.
\gls{jndi} wird überwiegend dafür verwendet, verteilte Java Applikationen miteinander interagieren zu lassen.

Eine Anwendung \textit{A} kann also einen Payload verarbeiten und das Resultat an einer remote location in Form eines Kompilats (Objekts) mithilfe von \gls{jndi} ablegen und serialisieren.
Eine andere Anwendung \textit{B} kann mit Hilfe eines \gls{jndi}-lookups Gebrauch von genau dieser Datei machen.

Hierzu muss man die zugrundeliegende Log4j Syntax verstehen:
%! Author = chaorn
%! Date = 06.01.23
\begin{lstlisting}[language=java, label=SampleLog.java]
import org.apache.logging.log4j.LogManager;
import org.apache.logging.log4j.Logger;
public class MyClass {
    private static final Logger LOGGER = LogManager.getLogger();
    // ...
       LOGGER.debug("Logging in user {} with birthday {}", user.getName(), user.getBirthdayCalendar());
}
\end{lstlisting}
\captionof{lstlisting}{\texttt{SampleLog.java}: Loggen mit Log4j}
\bigskip
%%%%%% !!!!!! mega whack. andere Formulierung !!!!!!

Das Loggen funktioniert mithilfe einer String interpolation. Der zur Laufzeit berechnete Parameter an erster Stelle wird hierbei automatisch
mithilfe einer String interpolation an der Stelle des ersten geschweiften Klammerpaars eingesetzt. Analog gilt diese Funktionsweise für den zweiten Parameter (siehe Listing 1).

Die Loggermethode \textit{error()} (siehe Listing 2) führt an dieser Stelle statt einer String interpolation zu einem von der Log4j Bibliothek bereits implementierten Lookup.
In diesem Fall handelt es sich um einen Sonderfall des Lookups, dem sogenannten \textit{\gls{jndi}-Lookup}. Der \gls{jndi}-Lookup wird von Log4j so interpretiert, dass eine Datei zur Laufzeit
an einer anderen Location bereitgestellt wird.
Durch den Lookup wird vom System in der \gls{jvm} das im Programm manipulierte Objekt selektiert und in den leeren geschweiften Klammern eingefügt.
Log4j bringt von sich aus \gls{jndi} Kompatibilität mit.

\newpage

Ein \gls{jndi}-lookup erfolgt wie folgt:
%! Author = chaorn
%! Date = 07.01.23

\begin{lstlisting}[language=java, label={lst:JNDILokupLog.java}]
import org.apache.logging.log4j.LogManager;
import org.apache.logging.log4j.Logger;
public class MyClass {
    private static final Logger LOGGER = LogManager.getLogger();
    // ...
       LOGGER.error("{}: ERROR {}", "${jndi:ldap://remote/file}", error.getMessage());
}
\end{lstlisting}
\captionof{lstlisting}{\texttt{JNDILookupLog.java}: JNDI-lookup}
\bigskip

Das besondere hierbei ist die Kennzeichnung des \gls{jndi}-lookups durch ein \textbf{\$} gefolgt von geschweiften Klammern.