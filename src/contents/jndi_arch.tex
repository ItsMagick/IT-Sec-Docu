%! Author = chaorn
%! Date = 06.01.23
\subsubsection{JNDI Architektur und Funktionsweise}
\newline
\begin{figure}[!htb] % ! - override default, h - place here, t - place figure at top of a page, b - place figure at bottom of a page
    \begin{center}
        \includegraphics[scale=0.75]{images/jndiarch}
    \end{center}
    \caption{JNDI Architektur}
\end{figure}
\newline\newline
Das \gls{jndi} besteht im Grunde aus einer \gls{api} und einem \gls{spi}. Der Programmierer einer Java Applikation interagiert überwiegend mit der \gls{api}, welche
naming und directory services zur Verfügung stellt. Für spezielle Funktionalitäten sorgt das \gls{spi}, dass je nach Bedarf ein benötigtes Modul ähnlich wie ein Pluginmanager
aktiviert.\footfullcite{JNDIArchitektur} Zu den bereitgestellten Services gehören unter anderem \gls{ldap} und \gls{rmi}.\\
