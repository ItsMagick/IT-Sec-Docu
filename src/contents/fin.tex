%! Author = charon
%! Date = 1/2/23

\section{Fazit}\label{sec:fazit}
Log4Shell wird von vielen Sicherheitsforschern als ``Schlimmste Sicherheitslücke des letzten Jahrzehnts''\footfullcite{guardianArticle} bezeichnet, und das wohl nicht zu Unrecht.
Bei CVE-2021-44228 kamen dabei mehrere Faktoren zusammen, die den perfekten Sturm auslösten.
Als Erstes war Log4Shell ein bereits aktiv ausgenutzter Zero-day in der am weitesten verbreiteten Logging Library für Java.
Außerdem hat Log4Shell eine Diskussion über Sicherheit in open-source Software gestartet, da die Sicherheitslücke 8 Jahre lang existiert hat, ohne entdeckt zu werden.

\subsection{Verwandte CVEs}\label{subsec:weiteres}
\begin{description}
    \item[CVE-2021-45046]\footfullcite{45046}\hfill \\Eine weitere Log4Shell Sicherheitslücke, die im offiziellen Patch für CVE-2021-4428\footfullcite{44228} übersehen wurde, wenn nicht-standard Konfigurationen verwendet werden.
    \item[CVE-2021-4104]\footfullcite{4104}\hfill \\Eine zu Log4Shell analoge \gls{rce} Schwachstelle in Log4j 1.2, die nur nicht-standard Konfigurationen betrifft.
    \item[CVE-2021-42550]\footfullcite{42550}\hfill \\Eine Sicherheitslücke im Nachfolger von Log4j 1.x, dem logback Framework, das Personen mit Zugriff auf die Konfigurationsdatei \gls{rce} Attacken über \gls{ldap} Server erlaubt.
    \item[CVE-2021-45105]\footfullcite{45105}\hfill \\Ein weiterer Teil der Log4Shell Sicherheitslücke, die \gls{dos} Attacken erlaubt.
\end{description}
