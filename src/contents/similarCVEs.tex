%! Author = opiwa
%! Date = 07.01.2023

\section{Verwandte Schwachstellen}\label{sec:similarCVEs}
Neben der Hauptschwachstelle CVE-2021-4428\footfullcite{44228} gehören zum Überbegriff Log4Shell noch weitere Schwachstellen, die entweder aus anderen Angriffsvektoren bestehen oder für andere Versionen von Log4j funktionieren.
Obwohl keine der anderen Log4Shell Schwachstellen einen \gls{cvss} Score von 10 aufweist, sind einige davon doch sehr schwerwiegend und teilweise auch nicht mit dem ersten Patch für Log4Shell mitigiert worden.

\begin{description}
    \item[CVE-2021-45046]\label{itm:cve-2021-45046}\footfullcite{45046}\hfill \\
    Eine weitere Log4Shell Sicherheitslücke, die im offiziellen Patch für CVE-2021-4428\footfullcite{44228} übersehen wurde.
    Erlaubt bei nicht-standard Konfigurationen weiterhin \gls{rce}.

    \item[CVE-2021-4104]\footfullcite{4104}\hfill \\
    Eine zu Log4Shell analoge \gls{rce} Schwachstelle in Log4j 1.2, die nur nicht-standard Konfigurationen betrifft.

    \item[CVE-2021-42550]\footfullcite{42550}\hfill \\
    Eine Sicherheitslücke im Nachfolger von Log4j 1.x, dem logback Framework, das Personen mit Zugriff auf die Konfigurationsdatei \gls{rce} Attacken über \gls{ldap} Server erlaubt.

    \item[CVE-2021-45105]\footfullcite{45105}\hfill \\
    Ein weiterer Teil der Log4Shell Sicherheitslücke, die durch einen StackOverflow ausgelöste \gls{dos} Attacken ermöglichen.
\end{description}
