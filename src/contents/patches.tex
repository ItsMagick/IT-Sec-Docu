%! Author = opiwa
%! Date = 04.01.2023

\subsection{Patches}\label{subsec:patches}
Da Log4j ein Bestandteil von vielen weiteren Libraries ist, ist es sehr wichtig überprüfen zu können, ob ein System angreifbar ist oder nicht.
Daher entstanden direkt nach der Disclosure die ersten Tools zum einfachen Testen, wie beispielsweise Log4ShellTools von Alexander Bakker.\footfullcite{log4ShellTools}

\subsubsection{Hotpatch von Apache Solr}


\subsubsection{erster offizieller Patch}
Drei Tage vor der öffentlichen Disclosure von Log4Shell, am 6.12.2021\footfullcite{log4jChange} veröffentlichte Apache den Patch 2.15.0.
Hier wird das message-lookup standardmäßig deaktiviert und \glsaccessshort{jndi} Verbindungen verwenden nun eine Whitelist, die standardmäßig nur localhost Adressen zulässt.\footfullcite{log4jSecurity}
Da dieser Patch weiterhin \gls{ldap} Verbindungen zulässt, ist CVE-2021-44228\footfullcite{44228} zwar offiziell geschlossen, man bleibt aber weiterhin für CVE-2021-45046\footfullcite{45046} verwundbar.

\subsubsection{weitere offizielle Patches}
Im am 13.12.2021\footfullcite{log4jChange} veröffentlichten Patch 2.16.0 wurde das message-lookup Feature komplett entfernt und \gls{jndi} standardmäßig deaktiviert.
Mit dem Patch 2.17.0 am 17.12.2021\footfullcite{log4jChange} wurde das \glsaccessshort{ldap} Protokoll aus \glsaccessshort{jndi}-Lookup entfernt und lässt nur noch das JAVA Protokoll zu.
