%! Author = opiwa
%! Date = 04.01.2023

\subsection{Patches}\label{subsec:patches}
Da Log4j ein Bestandteil von vielen weiteren Bibliotheken ist, ist es sehr wichtig überprüfen zu können, ob ein System angreifbar ist oder nicht.

Daher entstanden direkt nach der Disclosure die ersten Tools zum einfachen Testen, wie beispielsweise das Projekt Log4ShellTools von Alexander Bakker.\footcite{log4ShellTools}
Mit solchen Tools können SysAdmins schnell testen, ob Log4j möglicherweise durch verschachtelte Abhängigkeiten in einem Projekt implementiert wird, was insbesondere in großen Java Projekten einfach vorkommen kann.

\subsubsection{JNDI Live patch}
Mit Tools wie Log4jHotPatch\footcite{hotpatch} ist es möglich, eine verwundbare Instanz während des laufenden Betriebs zu patchen.
Hierzu wird die \verb|lookup()| Methode von \verb|org.apache.logging.log4j.core.lookup.JndiLookup| durch eine leere Version getauscht, wodurch \gls{jndi} Lookups ins Leere laufen.

Diese Art des Patches hat dadurch natürlich den Vorteil, dass das System nicht heruntergefahren werden muss, gilt aber nur als temporäre Lösung für das eigentliche Problem.
Daher bleibt als einzige permanente Lösung nur, die offiziellen Patches direkt von Apache zu verwenden.

\subsubsection{Erster offizieller Patch}
Drei Tage vor der öffentlichen Disclosure von Log4Shell, am 6.12.2021\footcite{log4jChange} veröffentlichte Apache den Patch 2.15.0.

Hier wird das message-lookup standardmäßig deaktiviert und \gls{jndi} Verbindungen verwenden nun eine Whitelist, die standardmäßig nur localhost Adressen zulässt.\footcite{log4jSecurity}
Da dieser Patch weiterhin \gls{ldap} Verbindungen zulässt, ist CVE-2021-44228\footcite{44228} zwar offiziell geschlossen, man bleibt aber weiterhin für CVE-2021-45046(siehe: \ref{itm:cve-2021-45046})\footcite{45046} verwundbar.

\subsubsection{Weitere offizielle Patches}
In dem am 13.12.2021\footcite{log4jChange} veröffentlichten Patch 2.16.0 wurde das message-lookup Feature komplett entfernt und \gls{jndi} standardmäßig deaktiviert.

Mit dem Patch 2.17.0 am 17.12.2021\footcite{log4jChange} wurde das \gls{ldap} Protokoll aus \gls{jndi}-Lookup entfernt und lässt nur noch das java Protokoll zu.

Damit ist die einzige Möglichkeit die Log4Shell Schwachstelle komplett zu patchen, Log4j 2.17.1(27.12.2021) oder neuer zu verwenden.
Somit waren vier Hotfixes in den achtzehn Tagen nach öffentlichem Disclosure nötig, bis Log4Shell von offizieller Seite komplett gepatcht wurde.
