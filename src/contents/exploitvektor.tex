%! Author = chaorn
%! Date = 07.01.23
\subsubsection{Exploitvektor}
Im Zusammenspiel von Logging und remote code mit \gls{ldap} versteckt sich bereits der Angriffsvektor. Was passiert nun, wenn ein Angreifer, statt einer validen remote location oder einer
validen logging Nachricht seinen eigenen Server als remote location injiziert?\\
Die einzige Voraussetzung um eine eigene remote location angeben zu können ist es, selber einen \gls{ldap}-Server aufzusetzen und zu hosten. Es gab 2016 bereits Ansätze
oder Verdacht zu einer ähnlichen Schwachstelle, die sich auch an der \gls{ldap}-Funktionalität bediente.\footcite{blackhatPresentation} Muñoz und Mirosh verfolgten jedoch einen anderen Ansatz und zwar den Ansatz des \gls{ldap} Entry Poisoning,
der dann zum CVE-2016-6497 führte.\footcite{cve2016}
