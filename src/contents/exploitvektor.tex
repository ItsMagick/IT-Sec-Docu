%! Author = chaorn
%! Date = 07.01.23
Im Zusammenspiel von Logging und remote code mit ldap versteckt sich bereits der Angriffsvektor. Was passiert nun, wenn ein Angreifer
seinen eigene remote location injiziert?\\
Die einzige Voraussetzung um eine eigene remote location angebenzukönnen ist selber einen \gls{ldap}-Server aufzusetzen und zu hosten. Es gab 2016 bereits Ansätze
oder Verdacht zu einer ähnlichen Schwachstelle, die sich auch an der \gls{ldap}-Funktionalität bediente.\footfullcite{blackhatPresentation} Muñoz und Mirosh verfolgten jedoch einen anderen Ansatz und zwar den Ansatz des \gls{ldap} Entry Poisoning,
der dann zum CVE-2016-6497 führte.\footfullcite{cve2016}
